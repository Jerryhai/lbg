% xecjk_preamble.tex
%%%%%%%%------------------------------------------------------------------------
%%%% xeCJK相关宏包

\usepackage{xltxtra,fontspec,xunicode}

%% \CJKsetecglue{\hskip 0.15em plus 0.05em minus 0.05em}
%% slanfont: 允许斜体
%% boldfont: 允许粗体
%% CJKnormalspaces: 仅忽略汉字之间的空白,但保留中英文之间的空白。
%% CJKchecksingle: 避免单个汉字单独占一行。
\usepackage[slantfont, boldfont]{xeCJK}

%% 针对中文进行断行
\XeTeXlinebreaklocale "zh"

%% 给予TeX断行一定自由度
\XeTeXlinebreakskip = 0pt plus 1pt minus 0.1pt

%%%% xeCJK设置结束
%%%%%%%%------------------------------------------------------------------------

%%%%%%%%------------------------------------------------------------------------
%%%% xeCJK字体设置

%% 设置中文标点样式,支持quanjiao、banjiao、kaiming等多种方式
\punctstyle{kaiming}

%% 设置缺省中文字体
\setCJKmainfont[BoldFont={Adobe Heiti Std},ItalicFont={Adobe Kaiti Std}]{Adobe Song Std}
%% 设置中文无衬线字体
\setCJKsansfont[BoldFont={Adobe Heiti Std}]{Adobe Kaiti Std}
%% 设置等宽字体
\setCJKmonofont{Adobe Heiti Std}

%% 英文衬线字体
\setmainfont{DejaVu Serif}
%% 英文等宽字体
\setmonofont{DejaVu Sans Mono}
%% 英文无衬线字体
\setsansfont{DejaVu Sans}

%% 定义新字体
\setCJKfamilyfont{song}{Adobe Song Std}
\setCJKfamilyfont{kai}{Adobe Kaiti Std}
\setCJKfamilyfont{hei}{Adobe Heiti Std}
\setCJKfamilyfont{fangsong}{Adobe Fangsong Std}
\setCJKfamilyfont{lisu}{LiSu}
\setCJKfamilyfont{youyuan}{YouYuan}

%% 自定义宋体
\newcommand{\song}{\CJKfamily{song}}
%% 自定义楷体
\newcommand{\kai}{\CJKfamily{kai}}
%% 自定义黑体
\newcommand{\hei}{\CJKfamily{hei}}
%% 自定义仿宋体
\newcommand{\fangsong}{\CJKfamily{fangsong}}
%% 自定义隶书
\newcommand{\lisu}{\CJKfamily{lisu}}
%% 自定义幼圆
\newcommand{\youyuan}{\CJKfamily{youyuan}}

%%%% xeCJK字体设置结束
%%%%%%%%------------------------------------------------------------------------

%%%%%%%%------------------------------------------------------------------------
%%%% 一些关于中文文档的重定义

%% 数学公式定理的重定义

\newtheorem{example}{例}
\newtheorem{algorithm}{算法}
%% 按section编号
\newtheorem{theorem}{定理}[section]
\newtheorem{definition}{定义}
\newtheorem{axiom}{公理}
\newtheorem{property}{性质}
\newtheorem{proposition}{命题}
\newtheorem{lemma}{引理}
\newtheorem{corollary}{推论}
\newtheorem{remark}{注解}
\newtheorem{condition}{条件}
\newtheorem{conclusion}{结论}
\newtheorem{assumption}{假设}

%% 章节等名称重定义
\renewcommand{\contentsname}{目录}
%\renewcommand{\abstractname}{摘要}
\renewcommand{\indexname}{索引}
\renewcommand{\listfigurename}{插图目录}
\renewcommand{\listtablename}{表格目录}
\renewcommand{\figurename}{图}
\renewcommand{\tablename}{表}
%\renewcommand{\appendixname}{附录}
%\renewcommand{\appendixpagename}{附录}
%\renewcommand{\appendixtocname}{附录}
%\renewcommand\refname{参考文献}

%% 设置chapter、section与subsection的格式
%\titleformat{\chapter}{\centering\huge}{第\thechapter{}章}{1em}{\textbf}
%\titleformat{\section}{\centering\sihao}{\thesection}{1em}{\textbf}
%\titleformat{\subsection}{\xiaosi}{\thesubsection}{1em}{\textbf}
%\titleformat{\subsubsection}{\xiaosi}{\thesubsubsection}{1em}{\textbf}

%%%% 中文重定义结束
%%%%%%%%------------------------------------------------------------------------

% en_preamble.tex
%%%%%%%%------------------------------------------------------------------------
%%%% 日常所用宏包

%% 控制页边距
\usepackage[top=2.5cm, bottom=2.5cm, left=2.5cm, right=2.5cm]{geometry}

%% 控制项目列表
\usepackage{enumerate}

%% 多栏显示
\usepackage{multicol}

%% hyperref宏包,生成可定位点击的超链接,并且会生成pdf书签
\usepackage[%
	pdfstartview=FitH,%
	CJKbookmarks=true,%
	bookmarks=true,%
	bookmarksnumbered=true,%
	bookmarksopen=true,%
	colorlinks=true,%
	citecolor=blue,%
	linkcolor=blue,%
	anchorcolor=green,%
	urlcolor=blue%
]{hyperref}

%% 控制标题
%\usepackage{titlesec}

%% 控制表格样式
\usepackage{booktabs}

%% 控制目录
%\usepackage{titletoc}

%% 控制字体大小
%\usepackage{type1cm}

%% 首行缩进,用\noindent取消某段缩进
\usepackage{indentfirst}

%% 支持彩色文本、底色、文本框等
\usepackage{color,xcolor}

%% AMS LaTeX宏包
\usepackage{amsmath}

%% 一些特殊符号
% \usepackage{bbding}

%% 支持引用
% \usepackage{cite}

%% LaTeX一些特殊符号宏包
% \usepackage{latexsym}

%% 数学公式中的黑斜体
% \usepackage{bm}

%% 调整公式字体大小:\mathsmaller, \mathlarger
% \usepackage{relsize}

%% 生成索引
% \makeindex

%%%% 基本插图方法
%% 图形宏包
\usepackage{graphicx}

%% 多个图形并排,参加lnotes.pdf
\usepackage{subfig}

% \begin{figure}[htbp] %% 控制插图位置
% \setlength{\abovecaptionskip}{0pt}
% \setlength{\belowcaptionskip}{10pt}
%% 控制图形和上下文的距离
% \centering %% 使图形居中显示
% \includegraphics[width=0.8\textwidth]{CTeXLive2008.jpg}
%% 控制图形显示宽度为0.8\textwidth
% \caption{CTeXLive2008安装过程} \label{fig:CTeXLive2008}
%% 图形题目和交叉引用标签
% \end{figure}
%%%% 基本插图方法结束

%%%% pgf/tikz绘图宏包设置
\usepackage{pgf,tikz}
\usetikzlibrary{shapes,automata,snakes,backgrounds,arrows}
\usetikzlibrary{mindmap}
%% 可以直接在latex文档中使用graphviz/dot语言,
%% 也可以用dot2tex工具将dot文件转换成tex文件再include进来
%% \usepackage[shell,pgf,outputdir={docgraphs/}]{dot2texi}
%%%% pgf/tikz设置结束


%%%% fancyhdr设置页眉页脚
%% 页眉页脚宏包
\usepackage{fancyhdr}

%% 页眉页脚风格
\pagestyle{plain}

%% 有时会出现\headheight too small的warning
\setlength{\headheight}{15pt}

%% 清空当前页眉页脚的默认设置
%\fancyhf{}
%%%% fancyhdr设置结束


%%%% 设置listings宏包用来粘贴源代码
%% 方便粘贴源代码,部分代码高亮功能
\usepackage{listings}

%% 所要粘贴代码的编程语言
\lstloadlanguages{}

%% 设置listings宏包的一些全局样式
%% 参考http://hi.baidu.com/shawpinlee/blog/item/9ec431cbae28e41cbe09e6e4.html
\lstset{
	showstringspaces=false, %% 设定是否显示代码之间的空格符号
	numbers=left, %% 在左边显示行号
	numberstyle=\tiny, %% 设定行号字体的大小
	basicstyle=\footnotesize, %% 设定字体大小\tiny, \small, \Large等等
	keywordstyle=\color{blue!70}, commentstyle=\color{red!50!green!50!blue!50},
	%% 关键字高亮
	frame=shadowbox, %% 给代码加框
	rulesepcolor=\color{red!20!green!20!blue!20},
	escapechar=`, %% 中文逃逸字符,用于中英混排
	xleftmargin=2em,xrightmargin=2em, aboveskip=1em,
	breaklines, %% 这条命令可以让LaTeX自动将长的代码行换行排版
	extendedchars=false %% 这一条命令可以解决代码跨页时,章节标题,页眉等汉字不显示的问题
}
%%%% listings宏包设置结束


%%%% 附录设置
%\usepackage[title,titletoc,header]{appendix}
%%%% 附录设置结束


%%%% 日常宏包设置结束
%%%%%%%%------------------------------------------------------------------------

%%%%%%%%------------------------------------------------------------------------
%%%% 英文字体设置结束
%% 这里可以加入自己的英文字体设置
%%%%%%%%------------------------------------------------------------------------

%%%%%%%%------------------------------------------------------------------------
%%%% 设置常用字体字号,与MS Word相对应

%% 一号, 1.4倍行距
\newcommand{\yihao}{\fontsize{26pt}{36pt}\selectfont}
%% 二号, 1.25倍行距
\newcommand{\erhao}{\fontsize{22pt}{28pt}\selectfont}
%% 小二, 单倍行距
\newcommand{\xiaoer}{\fontsize{18pt}{18pt}\selectfont}
%% 三号, 1.5倍行距
\newcommand{\sanhao}{\fontsize{16pt}{24pt}\selectfont}
%% 小三, 1.5倍行距
\newcommand{\xiaosan}{\fontsize{15pt}{22pt}\selectfont}
%% 四号, 1.5倍行距
\newcommand{\sihao}{\fontsize{14pt}{21pt}\selectfont}
%% 半四, 1.5倍行距
\newcommand{\bansi}{\fontsize{13pt}{19.5pt}\selectfont}
%% 小四, 1.5倍行距
\newcommand{\xiaosi}{\fontsize{12pt}{18pt}\selectfont}
%% 大五, 单倍行距
\newcommand{\dawu}{\fontsize{11pt}{11pt}\selectfont}
%% 五号, 单倍行距
\newcommand{\wuhao}{\fontsize{10.5pt}{10.5pt}\selectfont}
%%%%%%%%------------------------------------------------------------------------


%%%%%%%%------------------------------------------------------------------------
%%%% 一些个性设置

%% 设定页码方式,包括arabic、roman等方式
%% \pagenumbering{arabic}

%% 有时LaTeX无从断行,产生overfull的错误,这条命令降低LaTeX断行标准
%% \sloppy

%% 设定目录显示深度\tableofcontents
%% \setcounter{tocdepth}{2}
%% 设定\listoftables显示深度
%% \setcounter{lotdepth}{2}
%% 设定\listoffigures显示深度
%% \setcounter{lofdepth}{2}

%% 设定段间距
\setlength{\parskip}{0.5\baselineskip}

%% 设定行距
\linespread{1}

%% 中文破折号,据说来自清华模板
\newcommand{\pozhehao}{\kern0.3ex\rule[0.8ex]{2em}{0.1ex}\kern0.3ex}

%% 设定itemize环境item的符号
\renewcommand{\labelitemi}{$\bullet$}

%% 设定正文字体大小
% \renewcommand{\normalsize}{\sihao}

%%%% 个性设置结束
%%%%%%%%------------------------------------------------------------------------


%%%%%%%%------------------------------------------------------------------------
%%%% bibtex设置

%% 设定参考文献显示风格
\bibliographystyle{unsrt}

%%%% bibtex设置结束
%%%%%%%%------------------------------------------------------------------------

% appendix.tex
%%%%%%%%------------------------------------------------------------------------
%%%% 附录

% \appendix
% \appendixpage
%% 将附录条目添加到contents
% \addappheadtotoc

%%%% 附录结束
%%%%%%%%------------------------------------------------------------------------


% content.tex
