\section{格式化文本-字体形状和风格}
LaTeX的已经做了一些格式。例如,我们已经看到了部分的标题是
大于普通的文本和大胆的脸。现在,我们将学习如何修改外观
自己的文字。
	\subsection{Time for action调整字体形状}
我们将强调一个重要的词在文本中,我们将看到如何使文字出现
粗体,斜体或倾斜。我们将弄清楚如何突出词语的一部分,一些文字
这已经强调:
\item {创建新文档写入以下内容:}
	\begin{enumerater}
		\begin{lstlisting{[LaTeX]TeX}]
\documentclass{article}
\begin{document}
Text can be \emph{emphasized}.
Besides being \textit{italic} words could be \textbf{bold},
\textsl{slanted} or typeset in \textsc{Small Caps}.
Such commands can be \textit{\textbf{nested}}.
\emph{See how \emph{emphasizing} looks when nested.}
\end{document}

		\end{lstlisting}
	\end{enumerater}
	刚刚发生了什么?
	首先,我们使用命令\ EMPH,一个字作为参数传递给该命令。
	这个参数会以斜体字形状,因为这是默认的方式如何LaTeX的排版
	强调文本。


	文本格式的命令通常看起来像\ * {参数},其中**代表一个
	BF两个字母的缩写,如粗体,斜体,和SL倾斜。这个论点
	像我们所看到的将进行相应格式化。该命令后,随后的
	文字排版,因为它是前命令正是后右大括号
	结束标记的说法。我们检查出来。
	我们嵌套的命令\ textit和\ textbf,这使我们能够实现的
	那些样式和文字的组合都出现了斜体和粗体。


	大多数字体命令将显示同样的效果,如果他们被应用两次一样
	\ textbf {\ textbf {词}}的话会不会变得更大胆。 \ EMPH行为
	是不同的。我们已经看到,强调文本会显示为斜体,但如果我们用\ EMPH到
	再一块这段文字,它会改变斜体正常字体。想象一下,一个重要的
	定理完全以斜体排版,你应该还是有机会突出
	在这个定理的话。
	\ EMPH是所谓的语义标记,因为它的含义是指,不仅对
	文字的外观。
	%[
	两次强调,如粗体和斜体标记在同一
	时间,可能会被认为是有问题的风格。改变
	明智的和一贯的字体形状。
	%]

	\subsection{选择字体族}
	我们的例子中,你在本书中看到的标准字体的字体比较。虽然
	LaTeX的字体有装饰的外观,这本书的文字字体看起来简单,干净。
	另一种方式是不同的,在我们的代码例子:每封信都具有相同的宽度。让我们来看看
	我们如何能够实现在我们的作品。
	\subsection{Time for action-switching to sans-serif and to typewriter fonts}
	\textbackslash textsf : sans-serif font
	\textbackslash texttt : typewriter font
	\textbackslash textrm : Roman text - the default font with serifs.
\textbackslash textsf\{LaTeX\ resources on the internet\}
\textbackslash texttt\{http://www.ctan.org\}.
\textsf{LaTeX\ resources on the internet}
The best place for downloading LaTeX related software is CTAN.
Its address is \texttt{http://www.ctan.org}.
	\subsection{Time for action-switch to sans-serif and to typewriter fonts}
	\subsection{字体切换}
	\textbackslash sffamily
	\textbackslash rmfamily
	Command		Declaration		Meaning
	\textbackslash textrm\{\}	\textbackslash rmfamily
	\textbackslash textsf\{\}
	\textbackslash texttt\{\}
	\textbackslash textbf\{\}
	\textbackslash textmd\{\}
	\textbackslash textit\{\}
	\textbackslash textsl\{\}
	\textbackslash textsc\{\}
	\textbackslash textup\{\}
	\textbackslash textnormal\{\}
	\subsection{切换字体族}
	\subsection{字体命令和声明的小结}
	\subsection{划定命令的效果}
	\subsection{Time for action-exploring grouping by braces}
	\subsection{Time for action-探索字体大小}
	\subsection{Using environments}
	\subsection{Time for action-使用环境来调整字体大小}
